
\usepackage[utf8]{inputenc}
\usepackage[a4paper,
            bindingoffset=0.2cm,
            left=1cm,
            right=1cm,
            top=1cm,
            bottom=2cm,
            footskip=1cm]{geometry}

\usepackage{lipsum}
\usepackage[english]{babel}



\usepackage{graphicx} % Required to include images
\usepackage[labelfont=sc]{caption} %DT$18/02/2021 caption style
\usepackage{float} %DT$03/01/2020 better image positioning
\usepackage{cuted} %DT$29/07/2022 allows writing over the whole page

\usepackage{braket} % DT$13/11/2021 qm symbols
\usepackage{enumerate} % Custom item numbers for enumerations
\usepackage{placeins} %DT$04/01/2020 \FloatBarrier 
\usepackage{color} % Required for custom colors
\usepackage{amsmath,amsfonts,stmaryrd,amssymb,theorem} % Math packages
\usepackage{siunitx} %DT$03/01/2021 symbols ex:\SI{50}{\micro\second}
\sisetup{range-phrase={\text{\ -\ }},
     input-decimal-markers={.}, 
     range-units = single,
     output-decimal-marker = {.},
     group-digits=false}

\usepackage{booktabs}

\newcommand{\reffig}[1]{Figure~\ref{#1}}
\newcommand{\reftab}[1]{Table~\ref{#1}}
\newcommand{\refeqn}[1]{Equation~\ref{#1}}
\NewDocumentCommand\mat{mmmm}{%
\text{$\begin{pmatrix}#1 & #2\\#3 & #4\end{pmatrix}$}%
}

%%%%%%%%%%%%%%%%%%%%%%%%%%%%%%%%%%%%%%%%%%%% COMMENTS COLORS %%%%%%%%%%%%%%%%%%%%%%%%%%%%%%%%%%%%%%%%
\usepackage{dsfont}
\usepackage{xcolor}
\usepackage{todonotes}
\usepackage{ulem}
\usepackage{listings}

% global

\definecolor{todoxcolor}{HTML}{11aa00}
\newcommand{\todox}[2]{{\color{todoxcolor}\sout{#1}#2}}
\renewcommand{\todox}[1]{\todo[color=todoxcolor!20, inline]{{\it TODO:} #1}}

\definecolor{UniPDcolor}{HTML}{9B0014}
\newcommand{\UniPD}[1]{\todo[color=UniPDcolor, inline]{{\color{white}{\it UniPD contribution:} #1}}}
\newcommand{\todoUPD}[1]{\todo[color=UniPDcolor!20, inline]{{\it TODO UniPD:} #1}}


% people

\definecolor{fedecolor}{HTML}{FEDEBE}
\definecolor{fedecolor2}{HTML}{ff7600}
\newcommand{\fede}[2]{{\color{fedecolor2}{(Fede:\sout{#1} #2})}}
\newcommand{\fedecom}[1]{\todo[color=fedecolor!40, inline]{{\it Fede:} #1}}

%dark mode document
\pagecolor[rgb]{0,0,0} %black
\color[rgb]{0.5,0.5,0.5} %grey 


% hyperlinks
\usepackage{hyperref}
\hypersetup{
    colorlinks=true,
    linkcolor=blue,
    filecolor=magenta,      
    urlcolor=cyan,
    citecolor=.,
    pdftitle={Overleaf Example},
    pdfpagemode=FullScreen,
    }



% code

\usepackage{listings}

% colors for highlighting
\definecolor{codegreen}{rgb}{0,0.6,0}
\definecolor{codegray}{rgb}{0.5,0.5,0.5}
\definecolor{codepurple}{rgb}{0.58,0,0.82}
\definecolor{backcolour}{rgb}{0.95,0.95,0.92}
\definecolor{delim}{RGB}{20,105,176}
\definecolor{numb}{RGB}{106, 109, 32}
\definecolor{string}{rgb}{0.64,0.08,0.08}

% JSON
\colorlet{punct}{red!60!black}
\definecolor{background}{HTML}{EEEEEE}
\definecolor{delim}{RGB}{20,105,176}
\colorlet{numb}{magenta!60!black}


\lstdefinestyle{style}{
    backgroundcolor=\color{backcolour},   
    commentstyle=\color{codegreen},
    keywordstyle=\color{magenta},
    numberstyle=\tiny\color{codegray},
    stringstyle=\color{codepurple},
    basicstyle=\ttfamily\scriptsize,
    breakatwhitespace=false,
    breaklines=true,
    captionpos=b,
    keepspaces=false,
    numbers=left,
    numbersep=5pt,
    showspaces=false,
    showstringspaces=false,
    showtabs=false,
    tabsize=2
}
\lstset{style=style}

\lstdefinelanguage{json}{
    numbers=left,
    numberstyle=\tiny\color{codegray},,
    rulecolor=\color{black},
    showspaces=false,
    showtabs=false,
    breaklines=true,
    postbreak=\raisebox{0ex}[0ex][0ex]{\ensuremath{\color{gray}\hookrightarrow\space}},
    breakatwhitespace=true,
    basicstyle=\ttfamily\scriptsize,
    upquote=true,
    captionpos=b,
    numbersep=5pt,
    keepspaces=false,
    tabsize=2,
    showstringspaces=false,
    morestring=[b]",
    stringstyle=\color{string},
    literate=
     *{0}{{{\color{numb}0}}}{1}
      {1}{{{\color{numb}1}}}{1}
      {2}{{{\color{numb}2}}}{1}
      {3}{{{\color{numb}3}}}{1}
      {4}{{{\color{numb}4}}}{1}
      {5}{{{\color{numb}5}}}{1}
      {6}{{{\color{numb}6}}}{1}
      {7}{{{\color{numb}7}}}{1}
      {8}{{{\color{numb}8}}}{1}
      {9}{{{\color{numb}9}}}{1}
      {\{}{{{\color{delim}{\{}}}}{1}
      {\}}{{{\color{delim}{\}}}}}{1}
      {[}{{{\color{delim}{[}}}}{1}
      {]}{{{\color{delim}{]}}}}{1},
}


\newcommand{\aCapo}{ ~ \vspace{0.15cm} \\} %magari fare renewcommand di paragraph 