\section{Data Gathering}
Considering the context (US elections) it is imperative to have users data divided between left and right leaning, so a group of \textit{super-users} was selected using various sources. 
\textit{Super-users} are defined as follows:

\begin{itemize}
    \item Influencers: people whose notoriety is platform-built, without a background in institutions of entertainment;
    \item Politicians;
    \item Newspapers or news sites (i.e. The Washington Post).
\end{itemize}

The complete followers list, with relative sources (missing if selected arbitrarily by the authors), is the following:

\begin{itemize}
    \item Left-Wing: 
        \begin{itemize}
            \item @aocinthehouse \cite{wikiDem,yougovDem}
            \item @bernie \cite{wikiDem,yougovDem}
            \item @chrisdmowrey \cite{foxRight}
            \item @cnn \cite{biasCheck1,biasCheck2}
            \item @democracynow.org [unofficial]
            \item @genzforchange \cite{climateCulture}
            \item @ginadivittorio \cite{NYTpayInfluencers}
            \item @harryjsisson \cite{foxRight}
            \item @huffpost \cite{biasCheck1,biasCheck2}
            \item @msnbc \cite{biasCheck1,biasCheck2}
            \item @newyorker \cite{biasCheck1,biasCheck2}
            \item @nytimes \cite{biasCheck1,biasCheck2}
            \item @repbowman \cite{politico}
            \item @rynnstar \cite{https://doi.org/10.1002/poi3.287}
            \item @teamkennedy2024 \cite{wikiDem,yougovDem}
            \item @thedailybeast \cite{biasCheck1,biasCheck2}
            \item @underthedesknews \cite{foxRight}
            \item @vox \cite{biasCheck1,biasCheck2}
            \item @washingtonpost \cite{biasCheck1,biasCheck2}
        \end{itemize}
    \item Right-Wing:
        \begin{itemize}
            \item @alynicolee1126 \cite{mozilla}
            \item @babylonbee \cite{foxProFreeSpech}
            \item @clarksonlawson \cite{foxRight}
            \item @dailymail \cite{biasCheck1,biasCheck2}
            \item @dailywire \cite{biasCheck1,biasCheck2}
            \item @itsthemandrew \cite{mozilla}
            \item @notvictornieves \cite{foxRight}
            \item @real.benshapiro \cite{doi:10.1177/20563051231177938}
            \item @studentsforlife \cite{foxProFreeSpech}
            \item @theisabelbrown \cite{mozilla}
            \item @thesun \cite{biasCheck1,biasCheck2}
        \end{itemize}
\end{itemize}

Only five of each groups had been selected to data gathering, due to time constraints (\textit{@alynicolee1126, @babylonbee, @real.benshapiro, @clarksonlawson, @notvictornieves} for right-leaning, and \textit{@thedailybeast, @huffpost, @aocinthehouse, @repbowman, @newyorker} for left-leaning super-users). All data has been gathered using TikTok's official APIs (\url{https://developers.tiktok.com/doc/overview/})

\subsection{TikTok's APIs}



