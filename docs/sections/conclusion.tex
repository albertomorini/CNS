\section{Conclusions}

While gathering data through TikTok's official APIs has proven difficult, the amount of total information ultimately obtained was satisfying: 35,798 distinct followers and 182 videos were downloaded among 10 selected super-users.

This dataset was used to visualize the network and identify the presence of echo chambers (numerically confirmed by the low cosine similarity between super-users), measure users' engagement (revealing higher participation by right-leaning accounts), and to demonstrate the impact a posted video has on both engagement and the number of followers.

Finally the information was utilized to develop a prototype for inferring political orientation, although the high degree of isolation mentioned before made it difficult to draw clear conclusions. 
The model created can be used in different contexts, such as clime change or vaccines controversies.

All the source code can be found publicly at: \url{https://github.com/albertomorini/CNS}, while the data downloaded is kept secret to preserve the privacy of users.

In the end, the model created can go beyond politics, and can also be used to analyze different issues such as climate change denial, anti-vax controversial and so on.