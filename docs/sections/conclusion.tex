\section{Conclusions}
While gathering data through TikTok's official APIs has proven difficult, the amount of total information ultimately obtained was satisfying: 35,798 distinct followers and 182 videos were downloaded among 10 selected super-users.\\
This dataset was used to visualize the network and identify the presence of echo chambers, numerically confirmed by the low cosine similarity between super-users, measure users' engagement, revealing higher participation by right-leaning accounts, and to demonstrate the impact a posted video has on both engagement and the number of followers.\\
Finally the information was utilized to develop a prototype for inferring political orientation, although the high degree of isolation mentioned before made it difficult to draw clear conclusions.


\subsection{Extensions}
Regarding echo chambers, privacy inference, engagement, and content impact, better results could be achieved by significantly increasing the number of gathered data points. For example, one could circumvent the limitations imposed by TikTok's APIs by using third-party alternatives such as \url{https://github.com/davidteather/TikTok-Api}.\\
Another approach to enhance the social graph could involve collecting not only the super-users' followers but also the accounts they follow. This strategy, while powerful, would exponentially increase the number of data points, potentially making the processing phase rather computationally onerous.\\
Incorporating users' comments into the dataset would allow for sentiment analysis using large language models (LLMs), facilitating the study of polarization. These observations could be intersected with prior findings. Additionally, utilizing language recognition tools and analyzing users' pinned, shared, and liked videos (obtainable through official APIs) could help infer geo-location information, which would be valuable in our geopolitical context.