TikTok is the most prominent social network of 2024 boasting over 100 million users in the USA alone \cite{tktkStat}, a number that needs to be put into perspective by considering the actual demographic: 60\% of young adults (aged 18-24) and almost all kids (aged 5-15) use TikTok daily \cite{https://doi.org/10.1002/poi3.287}. Being this influential means not only to gather the attentions of advertisement companies, but also of politicians: the Democrat party began organizing paid influencers as early as 2020 (in the United States election 2020, presidential candidate Michael Bloomberg engaged in paid partnerships on social media \cite{10.3389/fcomm.2021.752656}), while Republican and Conservative hype houses also campaign on behalf of political candidates. Moreover in Germany, the political party CSU invited influencers to political events and since recently also creates influencer-like social media posts on platforms such as TikTok \cite{10.3389/fcomm.2021.752656}.\\
This also led to the creation of influencer-driven marketing firms, that now claim to control massive, immediately-deployable, stables of small-scale influencers on behalf of a given campaign. These “nano” and “micro” influencers aren't what most people think of when they imagine an influencer: these are everyday people who have captive, intimate social media audiences who identify with demographics especially appealing to U.S. political campaigns: Latinos in South Florida, Black voters in Atlanta and college-educated women in the Rust Belt \cite{theHilltktk}. 
It's also worth noting that a political influencer does not need to have an institutional background, in fact most of the times their notoriety and fame is platform-built \cite{doi:10.1177/20563051231177938}.\\
Despite all that, TikTok has tried to sell itself as a place for everything but politics. Since 2019, the company has banned paid political advertising on the site (“the nature of paid political ads is not something we believe fits the TikTok platform experience”), even tho a consistent amount of creators regularly use their platforms to disseminate political messaging and viewpoints and they don't always disclose their paid partnerships

62 percent of Republicans back a ban on the platform, while only 33 percent of Democrats feel the same

