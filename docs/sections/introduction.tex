\section*{Introduction}
TikTok is one of the most prominent social networks currently available, boasting over 100 million users in the USA alone\cite{tktkStat}. Approximately 60\% of young adults (aged 18-24) and nearly all children (aged 5-15) use TikTok daily\cite{https://doi.org/10.1002/poi3.287}. This significant influence attracts not only advertisers but also politicians: the Democratic Party began organizing paid influencers as early as the 2020 United States election (for instance, presidential candidate Michael Bloomberg engaged in paid partnerships on social media\cite{10.3389/fcomm.2021.752656}), while Republican and Conservative hype houses campaigned on behalf of political candidates. In Germany, the political party CSU invited influencers to political events and has recently started creating influencer-like social media posts on platforms such as TikTok\cite{10.3389/fcomm.2021.752656}.\\
This led to the creation of influencer-driven marketing firms, which now claim to control vast, immediately-deployable stables of small-scale influencers for various campaigns. These “nano” and “micro” influencers differ from the conventional image of an influencer: they are everyday people with captive, intimate social media audiences who represent demographics particularly appealing to U.S. political campaigns, such as Latinos in South Florida, Black voters in Atlanta, and college-educated women in the Rust Belt\cite{theHilltktk}. Political influencers often do not have an institutional background, in fact most of the times their notoriety and fame is platform-built\cite{doi:10.1177/20563051231177938}.\\
Despite this political engagement, TikTok has attempted to market itself as a platform for everything but politics: since 2019 the company has banned paid political advertising, stating that “the nature of paid political ads is not something we believe fits the TikTok platform experience.” Nevertheless, many creators regularly use the platform to disseminate political messages and viewpoints without disclosing whether the content is sponsored or not\cite{mozilla,politico}.\\
Given these dynamics, there is considerable value in studying user interactions on the platform. This work aims to do so, focusing on the context of the 2024 U.S. political election. The study will examine user movements using a social graph, analyze user similarity through cosine similarity, infer political affinity, measure engagement, and visualize the impact of publishing a video on followers and comments.